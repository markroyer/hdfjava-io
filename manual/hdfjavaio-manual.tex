
\documentclass{memoir}

% Jurabib must be included prior to hyperref package
\usepackage[titleformat=italic,titleformat=commasep,commabeforerest,%
   ibidem=strict,citefull=first,lookat,oxford,pages=format,idem]{jurabib}

\usepackage[T1]{fontenc}
\usepackage[margin=1in]{geometry}
\usepackage[pdftex]{graphicx}
\usepackage{multirow}
\usepackage{setspace}
\usepackage[usenames,dvipsnames,table]{xcolor}
\usepackage[colorlinks=true,linkcolor=blue,urlcolor=NavyBlue,citecolor=Fuchsia]{hyperref}
\usepackage{tabularx}
\usepackage[framemethod=TikZ]{mdframed}

\usepackage{forest}
\usetikzlibrary{arrows.meta}
\usetikzlibrary{backgrounds}

\usepackage{float}
%\floatstyle{boxed} 
\restylefloat{figure}

\usepackage{lstcustom}

%\pagestyle{plain}

\chapterstyle{ell}

\title{The \textit{HDFJavaIO} Library Documentation}
%\date{}
\author{Mark Royer}


%%%%%%%%%%%%%%%%%%%%%%%%%%%%%%%%%%%%%%%%%%%%%%%%%%%%%%%%%%%%%%%%%%%%
%%%%%%%%%%%%%%%%%%%%%%% STYLE FOR MAVEN %%%%%%%%%%%%%%%%%%%%%%%%%%%
%%%%%%%%%%%%%%%%%%%%%%%%%%%%%%%%%%%%%%%%%%%%%%%%%%%%%%%%%%%%%%%%%%%%


\definecolor{bashgray}{RGB}{249,249,249}
\definecolor{bashblue}{RGB}{47,111,171}
\mdfdefinestyle{mavencodebox}{%
 roundcorner=5pt,
 skipabove=10pt,skipbelow=10pt,
 middlelinewidth=1pt,
 backgroundcolor=bashgray,
 linecolor=bashgray,
 tikzsetting={draw=bashblue,line width=1pt,%
   dashed,%
   dash pattern= on 2pt off 2pt},
}
\lstnewenvironment{mavencode}[1][]
{\lstset{language=ant, % No maven support :(
    numbers=none,
    frame=,
    captionpos=b,
    belowskip=0pt,
    aboveskip=0pt,
    #1
  }
}{}
\surroundwithmdframed[style=mavencodebox]{mavencode}
        
%%%%%%%%%%%%%%%%%%%%%%%%%%%%%%%%%%%%%%%%%%%%%%%%%%%%%%%%%%%%%%%%%%%%
%%%%%%%%%%%%%%%%%%%%% END STYLE FOR MAVEN %%%%%%%%%%%%%%%%%%%%%%%%%
%%%%%%%%%%%%%%%%%%%%%%%%%%%%%%%%%%%%%%%%%%%%%%%%%%%%%%%%%%%%%%%%%%%%


\newcommand*{\note}[1]{
\begin{mdframed}[roundcorner=5pt,linewidth=.5pt,linecolor=red,backgroundcolor=red!10]
  \textbf{NOTE:} #1
\end{mdframed}
}

\newcommand*{\bera}[1]{\textbf{\fontfamily{beramono}\selectfont\ttfamily #1}}

\newenvironment*{berafont}{\fontfamily{beramono}\selectfont\ttfamily}{\par}

\newcommand*{\hdfc}{\bera{HDFJavaIO}}

\newcommand*{\java}[1]{{\color[rgb]{.5,0,.5}\bera{#1}}}
\newcommand*{\octave}[1]{{\color[rgb]{.5,0,0}\bera{#1}}}
\newcommand*{\hdf}[1]{{\color[rgb]{0,.5,0}\bera{#1}}}

\newcommand*{\ch}[1]{\multicolumn{1}{|c|}{\cellcolor{black!20}\textbf{#1}}}

% Tree is based on code from
% http://tex.stackexchange.com/questions/23647/drawing-a-directory-listing-a-la-the-tree-command-in-tikz
\forestset{
  dir tree/.style={
    for tree={
      parent anchor=south west,
      child anchor=west,
      anchor=mid west,
      inner ysep=1pt,
      grow'=0,
      align=left,
      edge path={
        \noexpand\path [draw, \forestoption{edge}] (!u.parent anchor) ++(1em,0) |- (.child anchor)\forestoption{edge label};
      },
      font=\sffamily,
      if n children=0{}{
        delay={
          prepend={[,phantom, calign with current]}
        }
      },
      fit=band,
      before computing xy={
        l=2em
      }
    },
  }
}

% Title page is based on examples from
% ftp://ftp.u-aizu.ac.jp/pub/tex/CTAN/documentation/latex-samples/TitlePages/titlepages.pdf

\newlength{\drop}% for my convenience
%% select a (FontSite) font by its font family ID
\newcommand*{\FSfont}[1]{\fontencoding{T1}\fontfamily{#1}\selectfont}

\newcommand*{\titleDB}{
\begingroup%
\FSfont{5pl} % FontSite URW Palladio (Palatino)
\drop = 0.14\textheight
\centering
\vspace*{\drop}
{\Large THE}\\[\baselineskip]
{\Huge The \textit{HDFJavaIO} Library}\\[\baselineskip]
{\Huge Documentation}\\[1.5\baselineskip]
{\LARGE \theauthor}\par
\vfill
    {\small\sffamily \href{https://umaine.edu}{The University of Maine}\\
      \href{https://umaine.edu/scis/}{School of Computing and Information Science}\\
      \today}\par
\endgroup
\thispagestyle{empty} % Removes title page number
}

\setsecnumdepth{subsubsection}
\setcounter{tocdepth}{2}

\begin{document}

\titleDB

\frontmatter

\tableofcontents*

\mainmatter

\chapter{Introduction}
\label{chap:intro}

\section{Overview}
\label{sec:overview}

Scientific Java programs commonly need to interact with high-level
programming languages that focus on numerical computations such as
Octave \cite{octave-url}.  This document discusses the \hdfc\ library that
allows Java programs to read and write Octave Hierarchical Data Format
5 (HDF5) \cite{hdf5-url} files.  The reference specification for the
Octave HDF5 file format is the Matlab 7.3 format \cite{mat-files-url}.

The goal of the \hdfc\ library is to facilitate the interaction of
Java programs with Octave.  HDF5 is a standard scientific file format
used for storing large data sets.  Octave can use HDF5 for reading and
storing data calculations.  The \hdfc\ library allows Java programs to
create HDF5 files in the Octave format without requiring the
programmer to know the details of the file structure.  The programmer
can simply call functions in the library, and the Java objects are
correctly written to an Octave HDF5 file in the correct structure.

\section{Data Forms}
\label{sec:data_forms}

This document refers to three forms of data: data that is in the Java
Programming Language, data that is represented in HDF5 files, and data
that is in the Octave Programming Language.  In an attempt to make it
clear which data form is being discussed, the following convention
will be used for types that appear in text.

Java types will appear in purple using the \java{beramono} font.  For
example, \java{boolean} is an example of a Boolean primitive data type
in Java.

Octave types will appear in maroon using the \octave{beramono} font.
For example, \octave{logical} is an example of the Boolean type in
Octave.

Finally, HDF5 file types will be represented using green
\hdf{beramono} font.  An example is the \hdf{64-bit floating-point},
which is used to represent Boolean types in HDF5 files.

Generally, the notation given above will be used throughout the
document; however, figures containing code examples will use
formatting and markup common for the specific language.

\chapter{Getting Started}
\label{chap:getting-started}

\section{Preparation \& Requirements}
\label{sec:preparation}

The \hdfc\ library is written in Java and therefore requires that a
JVM is installed.  The latest version of \hdfc\ has been compiled
using Java 8.  If you wish to make changes to the build so that the
library is compatible to older JVMs, see chapter \ref{chap:source} of
this manual.  Other requirements for the \hdfc\ library include the
HDFJava library, which can be found at the HDF Group website
\cite{hdf5-url}. Because the HDFJava library makes use of native code
and Java Native Interface (JNI), associated binaries (for each
platform) are required by the \hdfc\ library.

To overcome the difficulty of integrating a variety of
platform-specific binary files with Java programs, the \hdfc\ library
has been made available in multiple formats.  The first, and perhaps
simplest method, is for users of the Eclipse IDE, which can install an
\hdfc\ plug-in to manage dependencies.  Another option is to use the
\hdfc\ library with standard Java jar files.  This option, of course,
requires that the developer manage all required dependencies. Finally,
the \hdfc\ library can be used with Maven by inserting the dependency
into the required \bera{pom.xml}.  These options are explored more
deeply in the following subsections.

\subsection{Eclipse Plug-in}
\label{sec:eclipse-plug}

The library has been written as an OSGI (Eclipse plug-in), so the
easiest way to use the software is to add the following URL to the
list of available software sites in eclipse.

\begin{verbatim}
https://raw.githubusercontent.com/wiki/markroyer/hdfjavaio/edu.umaine.cs.hdfjavaio.repository/updates
\end{verbatim}

The above repository contains the dependencies for the
library. Alternatively, you may only want to add the following
composite repository, which contains a number of libraries that I find
useful when working with HDFJavaIO files.

\begin{verbatim}
https://rawgit.com/markroyer/p2-repositories/master
\end{verbatim}


\subsection{Jar Dependencies}
\label{sec:jar-dep}

The HDFJavaIO library can be directly downloaded from the following
links. Please note, however, that these binaries also require the
HDFJava library from the HDFGroup.
\\
\\
\begin{center}
\noindent
\begin{tabular}[h]{ | l | l | l | }
\hline
Version & Binary & Source \\
\hline
\hline
1.0 & \href{https://raw.githubusercontent.com/wiki/markroyer/hdfjavaio/edu.umaine.cs.hdfjavaio.repository/updates/1.0/1.0.0.v20160829-2316/plugins/edu.umaine.cs.hdfjavaio_1.0.0.v20160829-2316.jar}{hdfjavaio\_1.0.0.jar} & \href{https://raw.githubusercontent.com/wiki/markroyer/hdfjavaio/edu.umaine.cs.hdfjavaio.repository/updates/1.0/1.0.0.v20160829-2316/plugins/edu.umaine.cs.hdfjavaio.source_1.0.0.v20160829-2316.jar}{edu.umaine.cs.hdfjavaio.source\_1.0.0.jar} \\
\hline
\end{tabular}
  
\end{center}


\subsection{Maven with Tycho}
\label{sec:maven}

Maven can be used to help aleviate the dependency issues that might
arise while using HDFJava related libraries. If using maven, then the
required libraries can be added to the \bera{pom.xml} file as follows.

\begin{mavencode}
<repository>
  <id>royer</id>
  <layout>p2</layout>
  <url>https://rawgit.com/markroyer/p2-repositories/master</url>
</repository>
\end{mavencode}

TODO -- Add more details about using Tycho with Maven.

\section{Writing Octave HDF5 Files}
\label{sec:writing-octave-hdf5}

TODO

\section{Reading Octave HDF5 Files}
\label{sec:reading-octave-hdf5}

Reading Octave HDF5 files is the reverse process of writing them.  The
\hdfc\ library uses the file specification mentioned above to convert
the objects to the proper Java type.  There are a few caveats, though.

Iterable objects are written as \bera{matrix} or \bera{cell} types
based on the data that they contain.  Primitive data types are written
as \bera{matrix} types, and \java{String} data is written as
\bera{cell} type.  When these data types are read back into Java, they
are converted to Java arrays.  The mapping of the array data back to
the iterable object is not addressed.

Octave also contains additional unsigned primitive types.  When these
types are read by the \hdfc\ library, they are converted to the next
size up integer type.  For example, \bera{uint8} is converted to a Java
\java{short}.  The type \bera{uint64} is not handled at this
point; however, it's possible that this could be represented as a Java
\java{BigInteger}.

\chapter{Specification}
\label{chap:specification}

\section{Basic HDF5 File Format}
\label{sec:basics}

The \hdfc\ library uses Java Core Reflection to inspect Java
types and then properly write them to HDF5 files in the format
required by Octave.  These files will be referred to as Octave HDF5
files.

HDF5 files are structured in such a way that they resemble a file
system.  Specifically, HDF5 files are directed graphs, where data is
stored using b-trees. Using the file system notion, Octave HDF5 files
store variables in folders with the same name.  Each folder has an
attribute named \textit{OCTAVE\_NEW\_FORMAT} that is an \hdf{8-bit
  integer} specifying the Octave version.  Currently, all variables
are stored using version 1 by the \hdfc\ library.

Each variable ``folder'' contains a \textit{type} and \textit{value}.
The \textit{type} is a meta string describing the type of the data for
the variable.  The \textit{value} contains the actual data for the
variable.  Figure \ref{fig:simpleHDF5} is an example of an
Octave-formatted HDF5 file.  The file contains two Java variables
named \java{x} and \java{y}. The type of \java{x} is \java{boolean},
and its value is \java{true}.  The type of \java{y} is \java{char},
and its value is \java{a}.

In the HDF5 file is the Java variable \java{x} has a type string of
\hdf{bool}, and value \java{true} is stored as \hdf{1.0} which is a
\hdf{64-bit floating-point} type.  Similarly, the variable \java{y}
has a type string of \hdf{sq\_string}, and the value \java{a} is stored
as US-ASCII \hdf{8-bit} with the value \hdf{97}.

\begin{figure}[h!]
  \centering
\frame{
\begin{forest}
  dir tree
  [/
    [x (Attr : \textit{OCTAVE\_NEW\_FORMAT=1})
      [type : \textit{bool}
      ]
      [value : \textit{1.0}
      ]
    ]
    [y (Attr : \textit{OCTAVE\_NEW\_FORMAT=1})
      [type : \textit{sq\_string}
      ]
      [value : \textit{97}
      ]
    ]
  ]
\end{forest}
}
\caption{An example of an Octave-formatted HDF5 file containing two
  variables x and y.  Each variable has an attribute specifying the
  version, and it contains a type string along with the actual
  value.}\label{fig:simpleHDF5}
\end{figure}

\note{The use of the \textit{OCTAVE\_NEW\_FORMAT} attribute for each
  variable seems a little strange.  However, this attribute would
  allow you to mix versions of the variables, so I suppose it is a
  little more flexible than simply using one format for the entire
  file.  To my knowledge, currently there is only the 1 version.}

\section{Primitives}

As mentioned above, the HDF5 files contain folders for each variable.
Each folder contains a \textit{type} and a \textit{value} entry.
Table \ref{table:primitives} shows the values used for the
\textit{type} string and the actual HDF5 type used for storing the
\textit{value} of each Java primitive type.  Integer data types are
stored using little endian in the Octave formatted HDF5 file, so they
are converted to big endian in Java.  The table also shows what the
corresponding type becomes when it is read by Octave.

\begin{table}[h]
  \begin{center}
  \renewcommand\tabcolsep{6pt}
  \def\arraystretch{1.5}
  \begin{tabular}{ | l | l | l | l | }
    \hline
    \ch{Java type} & \ch{HDF5 type string} & \ch{HDF5 value type} & \ch{Octave type} \\
    \hline
    \hline
    \java{boolean} / \java{Boolean.class} & \hdf{bool} & \hdf{64-bit floating-point} & \octave{logical} \\
    \hline
    \java{char} / \java{Character.class} & \hdf{sq\_string} & \hdf{8-bit integer} & \octave{char} \\
    \hline
    \java{byte} / \java{Byte.class} & \hdf{int8 scalar} & \hdf{8-bit integer} & \octave{int8} \\
    \hline
    \java{short} / \java{Short.class} & \hdf{int16 scalar} & \hdf{16-bit integer} & \octave{int16} \\
    \hline
    \java{int} / \java{Integer.class} & \hdf{int32 scalar} & \hdf{32-bit integer} & \octave{int32} \\
    \hline
    \java{long} / \java{Long.class} & \hdf{int64 scalar} & \hdf{64-bit integer} & \octave{int64} \\
    \hline
    \java{float} / \java{Float.class} & \hdf{float scalar} & \hdf{32-bit floating-point} & \octave{single} \\
    \hline
    \java{double} / \java{Double.class} & \hdf{scalar} & \hdf{64-bit floating-point} & \octave{double} \\
    \hline
  \end{tabular}
  \end{center}
\caption{This table shows every Java primitive type, what its
  corresponding HDF5 type string and HDF5 value type is, and what its
  corresponding type is in Octave.}\label{table:primitives}
\end{table}

While most of the primitive values' corresponding types are not
surprising, the types for \java{boolean} and \java{char} should be
discussed a bit.  The \java{boolean} data type is stored as a
\hdf{64-bit floating-point} that can have the value of \hdf{0.0} or
\hdf{1.0}.  The type string for \java{char} is \hdf{sq\_string}
instead of the \bera{char} type which both languages use for
characters.

Another notable item is that Java has \java{Object} wrapper types that
represent each primitive, so these are represented in table
\ref{table:primitives}.  Java \java{Object} primitives are used widely
in the language and needed to be incorporated.  This is particularly
important because Java uses auto-boxing and unboxing when storing
primitive data in generic collections of objects, eg \java{List<E>},
\java{Set<E>}, \java{Collection<E>}, etc.

Since Java \java{Object} wrapper types need to be converted into
primitives, \java{null} values are not allowed.  If a user of the
\hdfc\ library has data collections that contain \java{null} values,
she will need to either fill these values with \java{NaN} (or some
other meaningful value), or the user can make use of \octave{cell}
discussed in section \ref{sec:cells}.

\note{Because Java uses the type-erasure paradigm for genericity, and
  because the \hdfc\ library uses reflection to determine the
  underlying type of the collection objects at runtime, this could
  lead to problems when dealing with collections that contain mixed
  types.  I will elaborate on this later.}

Octave also contains unsigned primitive types.  These types are
\octave{uint8}, \octave{uint16}, \octave{uint32}, and \octave{uint64}.
These data types are converted to their corresponding Java data type
with the same number of bits.  It is the responsibility of the user of
the \hdfc\ library to handle these appropriately.

For convenience, the \hdfc\ library contains utility functions to
upscale some unsigned types.  These conversions are shown in table
\ref{table:unsigned}.  Additional support for unsigned types has been
added to the \java{Integer} and \java{Long} classes in Java 8
\cite{oracle-datatypes-url}.

\begin{table}[h]
  \begin{center}
  \renewcommand\tabcolsep{6pt}
  \def\arraystretch{1.5}
  \begin{tabular}{ | l | l | l | l | l | }
    \hline
    \ch{Java type} & \ch{Translation} & \ch{HDF5 type string} & \ch{HDF5 value type} & \ch{Octave type} \\
    \hline
    \hline
    \java{short} & \java{(short)(v \& 0xff)} & \hdf{uint8 scalar} & \hdf{8-bit integer} & \octave{uint8} \\
    \hline
    \java{int} & \java{v \& 0xffff} & \hdf{uint16 scalar} & \hdf{16-bit integer} & \octave{uint16} \\
    \hline
    \java{long} & \java{v \& 0xffffffff} & \hdf{uint32 scalar} & \hdf{32-bit integer} & \octave{uint32} \\
    \hline
    \java{long}? & None? \java{BigInteger} class? & \hdf{uint64 scalar} & \hdf{64-bit integer} & \octave{uint64} \\
    \hline
  \end{tabular}
  \end{center}
  \caption{This table shows the upscaling done for unsigned primitives. The variable \java{v} represents the Java value read from the Octave-formatted HDF5 file with no bit modification.}\label{table:unsigned}
\end{table}


\section{Matrices}

Matrices are represented slightly differently in the Octave HDF5 file
format.  The details of which are shown in table \ref{table:matrices}.
One difference in arrays between Java and Octave is that Octave
requires array sizes to be rectangular (C-style arrays).  This is not
the requirement for Java which uses jagged arrays.  For jagged arrays,
make use of the \octave{cell} type discussed in section
\ref{sec:cells}.  Table \ref{table:matrices} is similar to the
primitive table, but it contains type information reported by the Java
Core Reflection mechanism, and it also contains matrix dimension and
size information.

An usual difference between table matrix table (\ref{table:matrices})
and the primitive table (\ref{table:primitives}) is the \java{boolean}
type.  As a scalar, the \java{boolean} is represented in the HDF5 file
as a \hdf{64-bit floating-point} number, but as a matrix,
\java{boolean} values are represented as \hdf{32-bit integer}s.  I am
not sure of the reason for the discrepancy.

\begin{table}
\begin{berafont}
  \begin{center}
  \renewcommand\tabcolsep{6pt}
  \def\arraystretch{1.5}
  \begin{tabular}{ | l | l | l | l | }
    \hline
    \ch{Java Array type} & \ch{HDF5 type string} & \ch{HDF5 value type} & \ch{Octave type} \\
    \hline
    \hline
    Z / Ljava.lang.Boolean; & \hdf{bool matrix} & \hdf{32-bit integer}, X & \octave{logical}, X \\
    \hline
    C / Ljava.lang.Character; & \hdf{sq\_string} & \hdf{8-bit integer}, X & \octave{char}, X \\
    \hline
    B / Ljava.lang.Byte; & \hdf{int8 matrix} & \hdf{8-bit integer}, X & \octave{int8}, X\\
    \hline
    S / Ljava.lang.Short; & \hdf{int16 matrix} & \hdf{16-bit integer}, X & \octave{int16}, X \\
    \hline
    I / Ljava.lang.Integer; & \hdf{int32 matrix} & \hdf{32-bit integer}, X & \octave{int32}, X \\
    \hline
    L / Ljava.lang.Long; & \hdf{int64 matrix} & \hdf{64-bit integer}, X & \octave{int64}, X \\
    \hline
    F / Ljava.lang.Float; & \hdf{float matrix} & \hdf{32-bit floating-point}, X & \octave{single}, X \\
    \hline
    D / Ljava.lang.Double; & \hdf{matrix} & \hdf{64-bit floating-point}, X & \octave{double}, X \\
    \hline
  \end{tabular}
  \end{center}
\end{berafont}
\caption{This table shows how every matrix composed of Java primitive
  types is represented in the HDF5 file, and what its corresponding
  type is in Octave. The \textbf{X} in columns 3 and 4 represent the number
  of dimensions and the size of each dimension (eg 3x3).}\label{table:matrices}
\end{table}

\section{String Data Types}
\label{sec:strings}

String data is handled slightly differently by the \hdfc\ library.
There are two reasons for this: Strings in Java are UTF-16 characters,
but they are ASCII character arrays in Octave; And \java{String}
objects in arrays are usually not rectangular in shape.

In order to write a Java String to an Octave HDF5 file, the string is
converted to an ASCII array, and then written to the file.  The same is
done to Java characters.

For multidimensional Java arrays to be written in the Octave HDF5 file
format, the strings must be made the same length so that the resulting
matrix is rectangular in shape. The resulting matrix uses empty
characters to pad strings in order to achieve the rectangular
shape. This is the default behavior in Octave when creating character
matrices.

If a padded rectangular character array is \textbf{not} what is
required by the developer (which is likely the case, since String
objects are likely to be of varying length in Java), then an
alternative approach is used.

In order to write Java arrays containing String objects of varying
length to an Octave HDF5 file, the Octave \octave{cell} structure is
used.  An example of this is given in section \ref{sec:cells}.

\section{Cell Data Types}
\label{sec:cells}

Cell data types are the Octave equivalent of Java \java{Object}
references.  Using the Octave \octave{cell} data type allows a user to
specify jagged arrays.

An example Octave HDF5 file with a two-dimensional Java \java{String}
array named \java{textCell} with the values
\lstinline${ {"one", "fine"}, {"autumn", "day"}}$ is shown in figure
\ref{fig:cellHDF5}.  Similar to other data types in the Octave-format
HDF5 file, the structure for the variable contains a \textit{type} and
\textit{value} field.  The \textit{type} meta string has the value of
\hdf{cell}.  The \textit{value}, however, is a sub-directory that
contains multiple sub-directories for the data.

\begin{figure}[h!]
  \centering
\frame{
\begin{forest}
  dir tree
  [/
    [textCell (Attr : \textit{OCTAVE\_NEW\_FORMAT=1})
      [type : \textit{cell}
      ]
      [value
        [\_0 (Attr : \textit{OCTAVE\_NEW\_FORMAT=1})
          [type : \textit{string}
          ]
          [value : \textit{[111,110,101]}
          ]
        ]
        [\_1 (Attr : \textit{OCTAVE\_NEW\_FORMAT=1})
        ]
        [\_2 (Attr : \textit{OCTAVE\_NEW\_FORMAT=1})
        ]
        [\_3 (Attr : \textit{OCTAVE\_NEW\_FORMAT=1})
        ]
        [dims : \textit{2x2}
        ]
      ]
    ]
  ]
\end{forest}
}
\caption{An example of an Octave formatted HDF5 file containing a 2x2
  \hdf{cell} structure with values \lstinline$\{ \{"one", "fine"\},
  \{"autumn", "day"\} \}$.  Each cell has a subfolder containing the
  actual data.  Jagged multidimensional cells can be achieved by
  nesting cell structures.  For brevity, only the first cell value has
  been expanded.}\label{fig:cellHDF5}
\end{figure}

\section{Complex Data Types}
\label{sec:complex}

Complex data types are represented in Octave and should be addressed.
There are two types: \octave{single complex} and \octave{double
  complex}.  These have not been addressed by the \hdfc\ library at this time.

\section{Iterable objects}
\label{sec:iterable}

Objects of the type \java{Iterable<T>} can also be written out to
Octave HDF5 file format using the \hdfc\ library.  Using the
\hdfc\ library, the standard method for writing Java objects is shown
in figure \ref{fig:standardWrite}.

Because of type erasure in Java, there is no way of ensuring that an
iterable collection contains the same types of objects.  For example,
a \java{List<Number>} may contain objects of type \java{Integer},
\java{Double}, \java{Float}, etc. The default write method can not be
used in a situation like this where the container is using mixed
types.  This is because it is not clear whether the type that should
be written to file should be \java{Integer}, \java{Double}, or
\java{Float}.

\begin{figure}[h!]
  \centering

\begin{lstlisting}[language=java,style=eclipse,numbers=none]
/**
 * Object is written to the file using the octave convention. This means
 * that a group will be created with the given label with two children. The
 * children are named "type" and "value". The type is a string containing
 * the octave type, and the value is object value.
 * 
 * @param file
 *            An open HDF5 file (Not null)
 * @param label
 *            The name of the object (Not null)
 * @param obj
 *            The value to write (Not null)
 * @throws H5Exception
 *             Thrown if there is an issue writing to the file (most likely
 *             a typing problem)
 */
public void writeObjectToFile(H5File file, String label, Object obj) throws H5Exception
\end{lstlisting}    

\caption{The standard object writing method used by the \hdfc\ library.}\label{fig:standardWrite}
\end{figure}


\note{In Java 5, the \java{Collections} class introduced a number
  of methods to address collections that required guaranteed type
  safety at runtime.  Some of these methods are
  \java{checkedCollection}, \java{checkedList},
  \java{checkedMap}, and others.  These methods check that the
  type specified at compile time is the same as the type inserted into
  the collection at run time.}

To write an iterable object, an alternative method is used that
contains an additional parameter that specify the type each
object will be converted to.  If an attempt to convert an object to the
given type fails, an \java{H5Exception} is thrown.
The method signature is shown in figure \ref{fig:iterableWrite}.

The method in figure \ref{fig:iterableWrite} is generic, which means
that it is more likely to pass type conversion at runtime.  This is
because the collection contents \bera{E} must be a subtype of Number
and \bera{E} is a supertype of the given type \bera{T}.

\note{The type bounding of \bera{E} in the method of figure
  \ref{fig:iterableWrite} to \bera{Number} is so that only number
  values can be written. But I'm not sure if this is necessary, since
  \java{char} and \java{String} objects should be fine.  I'm
  also not convinced that the type coercion is what I want.}

Let's consider an example iterable object of type \java{List<Double>}.  The only valid
value for the parameter \bera{type} would be \java{Class<Double>} since the class
\java{Double} is declared as final.

Now consider an iterable object of type \java{List<Number>}.
Valid types that could be provided are \java{Double},
\java{Integer}, \java{Float}, etc.  In this example, if the
type \java{Integer} is provided, then any floating point numbers
are truncated in the HDF5 file.  Conversely, if the type is \java{Double}, and the
list contains \java{Integer} types, a widening primitive
conversion occurs.

\begin{figure}[h!]
  \centering

\begin{lstlisting}[language=java,style=eclipse,numbers=none]
public <E extends Number, T extends E> void writeObjectToFile(H5File file, String label,
                        Iterable<E> it, T type) throws H5Exception
\end{lstlisting}    

\caption{The method used by the \hdfc\ library to write iterable data objects.}\label{fig:iterableWrite}
\end{figure}

Nested iterable objects have not been addressed, so don't do it!

\section{General Java Types}

This project has not addressed representing Java objects in the Octave
HDF5 format.  However, at first consideration, there appear to be two
approaches: Represent objects using Octave's c-style structs; or use
Octave's function objects.

If this project were to convert Java objects into structs, then the
method binding information would be lost.  However, the data
associated with the Java objects would be represented in the struct
object.  This seems reasonable for simple objects, eg
\java{Point2D.Float} or \java{Point3D.Float}.

An alternative approach would be to use the object-oriented function
approach.  A function is used as a constructor to create an object
similar to other functional languages, and then that function object
is passed as the first parameter to methods. An implementation using
this approach would require writing the object implementation to a
file that Octave could access.  Namespacing considerations would have
to be considered, too, since object definitions are globally defined
in Octave.

For now, this project does not address representing general Java
objects in HDF5 files.



\chapter{Related Libraries and Techniques}

There only a few projects (to the knowledge of this author) that allow
Java programmers to interact with Octave.  Three such projects are
\textit{Octave Java Library}, \textit{JavaOctave}, and \textit{joPAS}.

The \textit{Octave Java Library} is a standard package, included in
Octave since version 3.8, that allows Octave programs to invoke Java
from Octave.  Java libraries can be added by dynamically modifying the
classpath in the Octave environment.  However, the direction is the
wrong way. The JVM is invoked from a running Octave program instead of
Octave being invoked from a running JVM.

The \textit{joPAS} projects addresses running Octave from an active
JVM.  The project allows a Java programmer to invoke octave commands
through a simple API that passes string commands to a local Octave
process.  Unfortunately, the \textit{joPAS} project does not
support many Octave commands and types, and the project has become
dormant since 2005.

\textit{JavaOctave} is another Java library for invoking Octave
commands from a running Java program.  It is a slightly more robust
library that includes commands for converting Java types to an Octave
environment and executing Octave commands.  Similar to \textit{joPAS},
the \textit{JavaOctave} project has become dormant since 2012.  It
does have one update in June 2015 to run properly on Octave 3.8, but
it does not run properly on newer versions of Octave.

\chapter{Building the Source Code}
\label{chap:source}

The project easiest way to build the project using maven from the
edu.umaine.cs.hdfjavaio.parent directory. Typing

\begin{verbatim}
mvn clean verify
\end{verbatim}

will compile the project and create a repository containing all of the
related libraries in

\begin{verbatim}
../../hdfjavaio.wiki/
\end{verbatim}


\bibliography{hdfjavaio-manual}
\bibliographystyle{jurabib}



\end{document}


%%  LocalWords:  HDF JavaOctave joPAS classpath JVM API APIs blackbox
%%  LocalWords:  NCL NetCDF boolean dir Attr bool unboxing eg runtime
%%  LocalWords:  genericity UTF textCell subfolder Iterable iterable
%%  LocalWords:  writeObjectToFile checkedList qf hdfjavaio beramono
%%  LocalWords:  checkedCollection checkedMap subtype supertype uint
%%  LocalWords:  structs struct Namespacing BigInteger EOF initScript
%%  LocalWords:  Matlab TODO endian NaN xff xffff xffffffff upscaling
%  LocalWords:  HDFJava JVMs xml JNI OSGI HDFJavaIO
